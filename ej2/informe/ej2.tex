\documentclass[../../guia1.tex]{subfiles}

\begin{document}
\section*{Ejercicio 6}
\subsection*{a)}
Partiendo de la definición de la transformada z de una función:
\begin{equation}
H(z)=\sum_{n=-\infty}^{\infty} h(n) z^{-n}
\end{equation}

Si reemplazamos $z=e^{i2\pi f}$ en la ecuación, se obtiene :

\begin{equation}
H(z)|_{e^{i2\pi f}} = \sum_{n=-\infty}^{\infty} h(n)\left[ cos(n2\pi f) + i sin(n2\pi f) \right] \label{eq:zeta1}
\end{equation}

Realizando lo mismo con $H(z^{-1})$,

\begin{equation}
H(z^{-1})|_{e^{i2\pi f}} = \sum_{n=-\infty}^{\infty}h(n) \left[ cos(n2\pi f) - i sin(n2\pi f) \right] \label{eq:zeta2}
\end{equation}

Como se observa la equacion \ref{eq:zeta2} es la conjugada de \ref{eq:zeta1}, es decir

\begin{equation}
H(z)|_{e^{i2\pi f}} = \left[ H(z^{-1})|_{e^{i2\pi f}} \right]^* \label{eq:demo}
\end{equation}
Sabiendo que $f f^* =|f|^2 $, entonces por \ref{eq:demo} queda demostrado:

\begin{equation}
H(z)H(z^{-1})|_{e^{i2\pi f}}=|H(e^{i2\pi f})|^2
\end{equation}

\subsection*{b)}
Utilizando la propiedad demostrada el ejercicio anterior, se obtiene el modulo de la transferencia
\begin{equation}
H(z)H(z^{-1})=\frac{1 - a z + b z^2}{b - az + z^2} \frac{b - az + z^2}{1 - a z + b z^2} = 1 \Rightarrow |H(e^{i2\pi f})|^2 =1
\end{equation}


\subsection*{c)}
Los polos y ceros de la transferencia, son:

\begin{equation}
Cero=\frac{a \pm \sqrt{a^2-4b}}{2b}
\end{equation}

\begin{equation}
Polo=\frac{a \pm \sqrt{a^2-4b}}{2}
\end{equation}

Como a y b son reales, los polos y ceros pueden ser complejos conjugados.



\end{document}